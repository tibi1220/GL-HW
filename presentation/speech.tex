\documentclass[12pt]{article}
\usepackage[margin=2.5cm]{geometry}
\usepackage[magyar]{babel}
\usepackage{enumitem}
\usepackage{parskip}

\setlength{\parindent}{0pt}
\setlength{\parskip}{.66em}

\begin{document}

\section*{Célkitűzések}

A mai bemutatónk témája a gépi látás házi feladatunk, amelynek keretében magyar rendszámokat kellett automatikusan felismerni és kiolvasni.
A projekt célja tehát kettős volt:
\begin{itemize}[noitemsep]
	\item Detektálni az autók rendszámtábláját a képeken,
	\item Majd kiolvasni a rajtuk lévő karaktereket.
\end{itemize}

Ehhez egy közel 1400 képből álló adatbázis állt rendelkezésünkre, valamint egy .csv fájl, amely tartalmazta az egyes képekhez tartozó rendszámokat.

>>> NEXT SLIDE >>>

\section*{Projekt felépítése}

A projekt két fő részből áll:

\begin{itemize}[noitemsep]
	\item YOLOv8 modell segítségével detektáltuk a rendszámtáblákat a képeken.
	      \begin{itemize}
		      \item Rövidítés: You Only Look Once, azaz csak egyszer néz rá a képre.
		      \item A YOLOv8 egy mélytanulás alapú objektumdetektáló modell, amelyet a
		            képek gyors és pontos feldolgozására terveztek.
		      \item A YOLOv8 modell képes a képeken található objektumokat, például
		            autókat és rendszámtáblákat azonosítani.
	      \end{itemize}
	\item A detektált táblákból kivágott képeken OCR-t, pontosabban a \texttt{fast-plate-ocr} modellt használtuk a karakterek kiolvasásához.
\end{itemize}

A kész rendszert kétféleképpen lehet használni:
\begin{itemize}[noitemsep]
	\item Parancssoros módon, ahol egy teljes mappát lehet feldolgozni,
	\item Webes felületen keresztül, ahol a felhasználó egyesével tölthet fel képeket.
\end{itemize}

>>> NEXT SLIDE >>>

\section*{Képek kézi annotálása}

A munka első lépéseként a képek egy részét (kb 100 darabot) kézzel kellett annotálni – bejelölni, hogy hol található a rendszámtábla a képen.
Ez időigényes folyamat volt, de szükséges volt egy kezdeti tanítóadatbázis elkészítéséhez.
Az annotációt a YOLOv8 által használt formátumban kellett elmenteni, ami egy egyszerű szöveges fájl, amely tartalmazza a képen található objektumok helyét és típusát.

>>> NEXT SLIDE >>>

\section*{YOLO tanítása annotált képekkel}

A kézi annotáció után elindítottuk a YOLOv8 betanítását.
A tanításhoz augmentációt alkalmaztunk, beállítottuk a tanulási rátát, súlycsökkentést, és más hiperparamétereket is.

\begin{itemize}[noitemsep]
	\item Augmentáció: a képek véletlenszerű elforgatása, tükrözése, méretezése és színkorrekciója.
	\item Tanulási ráta: a tanulás sebessége, amely befolyásolja, hogy a modell mennyire gyorsan tanulja meg a mintákat.
	\item Súlycsökkentés: a modell súlyainak csökkentése a túltanulás elkerülése érdekében.
	\item hiperparaméterek: a modell teljesítményét befolyásoló paraméterek, amelyeket a tanítás előtt kell beállítani.
	\item Batch size: a tanítás során egyszerre feldolgozott képek száma.
	\item Epoch: a tanítási ciklusok száma, amely alatt a modell többször is látja az összes képet.
\end{itemize}

>>> NEXT SLIDE >>>

\section*{Automatikus annotálás}

Miután elkészült a kezdeti YOLO modell, már képesek voltunk a többi, még nem annotált képet automatikusan feldolgozni.
A detekciós küszöböt 0.5-re állítottuk, hogy kiszűrjük a gyengébb találatokat.
Ezzel jelentős időt spóroltunk az annotálás során.

>>> NEXT SLIDE >>>

\section*{Annotációk ellenőrzése}

Az automatikus annotálás nem tökéletes, ezért kézzel átnéztük a generált jelöléseket.
Például volt, amikor a modell nem jelölt meg semmit, vagy rossz helyre rajzolt keretet.
Ezeket manuálisan javítottuk.

>>> NEXT SLIDE >>>

\section*{Végső tanítás}

Miután elkészült a teljes, ellenőrzött annotációs állomány, újra betanítottuk a modellt.
Ez adta a legjobb pontosságot, így már megbízhatóan detektálta a rendszámtáblákat.
A felhasznált paraméterek az első tanítással megegyezőek voltak.

>>> NEXT SLIDE >>>

\section*{OCR működése}

Ezután következett az OCR modell. A fast-plate-ocr kifejezetten rendszámtáblák karaktereinek felismerésére lett optimalizálva.
A YOLO által kivágott rendszámtábla kép bekerül a modell bemenetére, ahol először szürkeárnyalatosra konvertáljuk, kontrasztot javítunk, majd átméretezzük.
A kimenet egy karakterlánc, amely a rendszám betűit és számjegyeit tartalmazza.
Emellett egy bizonytalansági értéket is ad, ami jelzi, mennyire magabiztos a modell az adott predikcióban.

>>> NEXT SLIDE >>>

\section*{Mappás tesztelés}

A fejlesztett rendszerrel képesek voltunk egy mappában lévő összes képet feldolgozni.
A bal oldalon néhány példa látható, ahol sikeresen detektált rendszámtáblákat.
A jobb oldali oszlopban pedig láthatjuk, hogy a rendszer egy .csv fájlban sorolja fel a felismert rendszámokat – ez már teljesen automatizált módon történik.

>>> NEXT SLIDE >>>

\section*{Webes felület}

Készítettünk egy egyszerű webes alkalmazást is.
A felhasználó képet tölthet fel, a rendszer felismeri a rendszámot, és visszaadja az eredményt.
A frontend React, a backend FastAPI alapú.

>>> LIVE DEMO >>>

>>> NEXT SLIDE >>>

\section*{Zárás}

Összefoglalva: sikeresen építettünk egy rendszert, amely magyar rendszámtáblákat képes detektálni és felismerni.
A YOLOv8 pontos detekciókat adott, az OCR modell pedig megbízható karakterfelismerést.
A rendszer parancssoros és webes formában is működik.

Köszönjük a figyelmet!

\end{document}
